\section{Testing and Validation}
To ensure the correct operation of the emulated system, we developed multiple applications that implement and test functionalities related to all the modifications we made in QEMU.


\subsection{Test Peripheral Initialization}

\begin{itemize}
    \item \textbf{UART:} Configuration of a UART peripheral for data transmission and reception.
    \item \textbf{CAN:} Configuration of a CAN peripheral, with two clients attempting to connect to the CAN bus to simulate communication between devices.
\end{itemize}


\subsection{Test Peripheral Functions Implementation}

\paragraph{UART Test}

\begin{itemize}
    \item Data transmission and reception via UART were implemented to verify that data is correctly exchanged between the microcontroller and an external device.
    \item Transmission errors were emulated by altering control bits, such as simulating a parity error, to verify the correct error handling mechanism.
\end{itemize}

\paragraph{CAN Test}

\begin{itemize}
    \item Data transmission and reception on the CAN bus were implemented, with two clients attempting to interact with the bus to simulate real traffic.
    \item CAN transmission errors were emulated, such as packet loss or data corruption, by modifying control bits.
\end{itemize}


\subsection{FreeRTOS Functionality Verification}
To verify the correct operation of FreeRTOS, the application was compiled using the NXP toolchain within the designated IDE. Additionally, several tasks were added, managed by a semaphore, to test the system's multitasking capabilities.

